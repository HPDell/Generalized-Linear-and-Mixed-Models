\documentclass[UTF8]{ctexart}

\usepackage{unicode-math}
\usepackage{amsmath}
\setmainfont{XITS}
\setmathfont{XITS Math}
\usepackage{mathshortcuts}

\usepackage[left=2.1cm,right=2.1cm,top=1.9cm,bottom=1.9cm]{geometry}

\title{\textbf{线性混合模型笔记}\\\large 参考 \textit{Generalized, Linear and Mixed Models}}
\author{\textit{胡奕公}}
\date{}

\begin{document}
\maketitle
\tableofcontents

\section{线性混合模型}

线性模型(Linear Model, LM)的出发点是 $ E[\vy] = \vX\vbeta $,其中 $ \vbeta $ 是固定效应。线性混合模型(Linear Mixed Model, LMM)在线性模型的基础上,添加了 $ \vZ\vu $,其中 $ \vZ $ 和 $ \vX $ 类似,是已知的数据矩阵;$ \vu $ 是随机效应。尽管 $ \vu $ 中的元素是随机变量,但是在其取值的条件下确定模型是非常方便的,即
\begin{equation}\label{equ:lmm-exp-cond-mean}
    E[\vy|\vu] = \vX\vbeta + \vZ\vu
\end{equation}
即对于任意一个确定的 $ \vu $,式\eqref{equ:lmm-exp-cond-mean}是一个条件均值\footnote{如果使用 $\vU$ 代表随机变量,$\vu$ 代表实际值,则需要将 $E[\vy|\vu]$ 改写为 $E[\vy|\vU=\vu]$,但这样的书写方式比较繁琐,因此保留 $E[\vy|\vu]$ 的写法}。对于 $ \vu $,认为其满足 $ \vu \sim (\vzero, \vD) $,即 $ E[\vu] = \vzero, \var(\vu) = \vD $,这不会有任何损失\footnote{原文中有证明}。

为了确定 $ \var(\vy) $,根据 $ \var(\vu) = \vD $,定义
\begin{equation}
    \var(\vy|\vu) = \vR
\end{equation}
又有 $ E[\vu] = 0 $,则
\begin{equation}\label{equ:y-mean-var}
    \vy \sim (\vX\vbeta, \vZ\vD\vZ\T+\vR)
\end{equation}
可以发现,固定效应只影响均值,随机效应只影响方差。

\section{因变量方差的结构特征}

根据式\eqref{equ:y-mean-var},$ \var(\vy) $ 的表达式是
\begin{equation}
    \vV = \var(\vy) = \vZ\vD\vZ\T + \vR
\end{equation}
通过下面的一些例子,可以对其进行简化。

\subsection{案例}

假设有一组关于学生数学成绩的数据,包括纽约15个高中、4个九年级班的所有学生。除了男生和女生之间的差异(这需要通过固定效应建模),毫无疑问,有三种变异的来源:学校之间、同学校班级之间、同班级不同学生之间。令学校 $ i $、班级 $ j $、学生 $ k $ (性别 $ t $)的成绩是 $ y_{tijk} $,则模型表达式可以是
\begin{equation}
    E[y_{tijk}|s_i,c_{ij}] = \beta_t + s_i + c_{ij}
\end{equation}
其中 $ \beta_t $ 是性别 $ t $ 产生的固定效应。学校的影响(即 $ s_i(i=1, 2, \cdots, 15) $ 和班级的影响(即 $ c_{ij}(j=1, 2, \cdots, 4) $)被认为是随机效应。使用 $ p_{tijk} $ 代表每个学生成绩中所有无法被 $ \beta_t, s_i, c_{ij} $ 解释的部分。因此式\eqref{equ:lmm-exp-cond-mean}中 $ \vX\vbeta $ 的 $ \vbeta $ 会具有两个元素,分别是代表男性的 $ \beta_m $ 和代表女性的 $ \beta_f $。同时,$ \vZ\vu $ 中的 $ \vu $ 会有 15 个 $ s_i $ 效应以及 60 个 $ c_{ij} $ 效应。

\subsection{效应间的协方差}

\end{document}
